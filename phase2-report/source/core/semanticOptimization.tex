Remove constraints specified on the database which are semantically or statistically guaranteed to happen, as it adds unneeded overhead without any use.
\subsubsection{Semantic Constraints}
Constraints that are guaranteed to happen based on the physical meaning of the database, like:
\begin{itemize}
     \item our database is built on Egypt's Disasters, there's no 'Others' gender in our community, only males and females, so no need to specify that
     \begin{verbatim}
     Person.gender IN (0,1)
     \end{verbatim}
     
     \item If we need to retrieve all not recent Incidents that happened before 5 years at least and are reviewed by not recent government representatives that started their job before 5 years at least, no need to mention both, as for sure if an incident happened say 6 years ago, it's guaranteed that its government representative reviewed it 6 years ago, so it's guaranteed that he/she is not recent as well.
     Use:
     \begin{verbatim}
     Incident.year > 5
     \end{verbatim}
     Instead of:
     \begin{verbatim}
     YEAR(Government_Representative.date_of_join) > 5 and Incident.year > 5
     \end{verbatim}
     
\end{itemize}



\subsubsection{Statistic Constraints}
Constraints that are guaranteed to happen based on the current statistics of the database, like:

\begin{itemize}
     \item all Government Representatives are older than 20 years old in the database statistics, so no need for
     \begin{verbatim}
     Government_Representative.age >= 20
     \end{verbatim}
     
     \item all Citizens have trust level up to 10 in the database statistics, so no need for
     \begin{verbatim}
     Citizen.trust_level <= 10
     \end{verbatim}
     
\end{itemize}