In this work, we have discussed various database optimization techniques and showed how the query execution time can be affected for both \emph{SQL} and \emph{NoSQL} databases. The final remarks can be summarized as follows :
\begin{itemize}
    \item With good optimization, \emph{SQL} databases can achieve a comparable performance with \emph{NoSQL} databases.
    \item \textbf{MySQL} is a very versatile \emph{DBMS} that can adapt to multiple optimization operations, enabling the user to increase queries execution speed.
    \item \textbf{MongoDB} can be the perfect choice for a \emph{NoSQL DBMS}, as it's easy to use and deploy on very large systems.
    \item For some operations, \emph{index optimization} can provide a huge performance improvement, if it's done right on specific fields in the database.
    \item It's a good practice to extract \emph{semantic} and \emph{statistical} heuristics based on your database. This can provide the developer with good insights on how to optimize queries and eliminate redundant operations.
    \item Try to keep the record size in frequently-accessed tables \emph{fixed}, as it's much faster to access fixed-size records. Moreover, it's better to partition the infrequent variable-size fields into separate tables.
\end{itemize}